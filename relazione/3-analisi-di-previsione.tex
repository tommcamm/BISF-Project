%%% CAPITOLO 3

\section{Analisi di previsione}

Per la costruzione del modello di previsione ho deciso di utilizzare il modello di analisi statistica chiamato
\textbf{ARIMA}\footnote{
    \href{https://www.investopedia.com/terms/a/autoregressive-integrated-moving-average-arima.asp}{https://www.investopedia.com/terms/a/autoregressive-integrated-moving-average-arima.asp}
} (\textbf{A}uto\textbf{R}egressive \textbf{I}ntegrated \textbf{M}oving \textbf{A}verage).
Questo modello utilizza i dati delle serie temporali per permettere di capire meglio il dataset e per effettuare predizioni su trend futuri.

Questo sistema è una forma di \emph{analisi di regressione} che calcola la forza di una variabile aleatoria indipendente relativamente ai cambiamenti di altre variabili.
L'obiettivo di questo modello è di effettuare predizioni sul futuro delle securities esaminando le differenze tra i valori nelle serie invece di utilizzare valori attuali.  

Possiamo evidenziare le componenti fondamentali di ARIMA come:
\begin{itemize}
    \item \emph{Autoregression (AR)}: Si Riferisce ad un modello che mostra una variabile che cambia
\end{itemize} 