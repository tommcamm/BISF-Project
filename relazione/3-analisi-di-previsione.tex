%%% CAPITOLO 3

\section{Analisi di previsione}

Per la costruzione del modello di previsione ho deciso di utilizzare il modello di analisi statistica chiamato
\textbf{ARIMA}\footnote{
    \href{https://www.investopedia.com/terms/a/autoregressive-integrated-moving-average-arima.asp}{https://www.investopedia.com/terms/a/autoregressive-integrated-moving-average-arima.asp}
} (\textbf{A}uto\textbf{R}egressive \textbf{I}ntegrated \textbf{M}oving \textbf{A}verage).
Questo modello utilizza i dati delle serie temporali per permettere di capire meglio il dataset e per effettuare predizioni su trend futuri.

Questo sistema è una forma di \emph{analisi di regressione} che calcola la forza di una variabile aleatoria indipendente relativamente ai cambiamenti di altre variabili.
L'obiettivo di questo modello è di effettuare predizioni sul futuro delle securities esaminando le differenze tra i valori nelle serie temporali invece di utilizzare valori attuali.  

Possiamo evidenziare le componenti fondamentali di ARIMA scomponendo il suo acronimo:
\begin{itemize}
    \item \emph{Autoregression (AR)}: Si Riferisce ad un modello che mostra una variabile che cambia in base alla regressione
    del suo passato o dei suoi valori precedenti.
    \item \emph{Integrated (I)}: Rappresenta la differenziazione di osservazioni grezze per permettere alla serie temporale di diventare
    stazionaria. (permettendo la applicazione della auto regressione e media mobile \emph{ARMA}).
    \item \emph{Moving average (MA)}: Incorpora la dipendeza tra una osservazione ed l'errore residuo dal modelli di media 
    mobile applicato alle osservazioni passate.
\end{itemize}

Nel modello ARIMA ogni componente visto qui sopra funziona come parametro avente una notazione standard, 
un esempio di notazione standard sarebbe ARIMA con \(p, d\) e \(q\), dove valori interi sostituiscono i parametri
per indicare la tipologia del modello ARIMA utilizzato. I parametri posso essere definiti come:

\begin{itemize}
    \item \(p\): Il numero di osservazioni passate nel modello.
    \item \(d\): Il numero di volte che le osservazioni grezze sono state differenziate, chiamato anche grado di differenziazione.
    \item \(q\): la dimensione della finestra relativa alla media mobile, chiamato anche ordine della media mobile.
\end{itemize}

Impostando i parametri visti qui sopra possiamo ottenere dei \emph{casi particolari}, utili per le analisi:

\begin{itemize}
    \item \(\text{ARIMA} (0,0,0)\): Rumore bianco.
    \item \(\text{ARIMA} (0,1,0)\) senza costante: Passeggiata 
    aleatoria\footnote{
        Serie con passi in direzioni casuali, \href{https://it.wikipedia.org/wiki/Passeggiata_aleatoria}{https://it.wikipedia.org/wiki/Passeggiata\_aleatoria}
    }.
    \item \(\text{ARIMA} (p,0,q)\): \(\text{ARMA} (p,q)\).
    \item \(\text{ARIMA} (p,0,0)\): modello \(\text{AR} (p)\).
    \item \(\text{ARIMA} (0,0,q)\): modello \(\text{MA} (q)\).
    \item \(\text{ARIMA} (0,1,2)\): modello "smorzato" di 
    Holt.\footnote{
        \href{https://otexts.com/fpp3/holt.html}{https://otexts.com/fpp3/holt.html}
    }
    \item \(\text{ARIMA} (0,1,1)\): senza costante: modello SES.
    \item \(\text{ARIMA} (0,2,2)\): metodo lineare di Holt\footnotemark[28], con errori aggiuntivi.
\end{itemize}

Una delle più note debolezze del modello ARIMA nel contesto finanziario è la inabilità di catturare il clustering della volatilità
che si osserva nella maggior parte degli asset finanziari.

Per questo progetto al fine di identificare i parametri migliori \(p, d \text{ e } q \) da applicare è stata utilizzata la libreria 
\emph{pmdarima}\footnote{
    \href{http://alkaline-ml.com/pmdarima/1.8.3/about.html}{http://alkaline-ml.com/pmdarima/1.8.3/about.html}
}, che grazie alla funzione \verb|auto_arima| mediante una ricerca a griglia riesce a trovare valori che minimizzano il valore 
\verb|AIC|\footnote{
    Stimatore per gli errori di predizione, \href{https://en.wikipedia.org/wiki/Akaike_information_criterion}{https://en.wikipedia.org/wiki/Akaike\_information\_criterion}
}.

\subsection{Modello di previsione per Meta (FB)}

Per Meta, è stato identificato come valore dei parametri migliore \(p=3, d=1, q=1\) con un errore \verb|AIC| pari a \(1836.767\) (figura x).