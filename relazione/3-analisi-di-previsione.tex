%%% CAPITOLO 3

\section{Analisi di previsione}

Per la costruzione del modello di previsione ho deciso di utilizzare il modello di analisi statistica chiamato
\textbf{ARIMA}\footnote{
    \href{https://www.investopedia.com/terms/a/autoregressive-integrated-moving-average-arima.asp}{https://www.investopedia.com/terms/a/autoregressive-integrated-moving-average-arima.asp}
} (\textbf{A}uto\textbf{R}egressive \textbf{I}ntegrated \textbf{M}oving \textbf{A}verage).
Questo modello utilizza i dati delle serie temporali per permettere di capire meglio il dataset e per effettuare predizioni su trend futuri.

Questo sistema è una forma di \emph{analisi di regressione} che calcola la forza di una variabile aleatoria indipendente relativamente ai cambiamenti di altre variabili.
L'obiettivo di questo modello è di effettuare predizioni sul futuro delle securities esaminando le differenze tra i valori nelle serie temporali invece di utilizzare valori attuali.  

Possiamo evidenziare le componenti fondamentali di ARIMA scomponendo il suo acronimo:
\begin{itemize}
    \item \emph{Autoregression (AR)}: Si Riferisce ad un modello che mostra una variabile che cambia in base alla regressione
    del suo passato o dei suoi valori precedenti.
    \item \emph{Integrated (I)}: Rappresenta la differenziazione di osservazioni grezze per permettere alla serie temporale di diventare
    stazionaria.
    \item \emph{Moving average (MA)}: Incorpora la dipendeza tra una osservazione ed l'errore residuo dal modelli di media 
    mobile applicato alle osservazioni passate.
\end{itemize}

Nel modello ARIMA ogni componente visto qui sopra funziona come parametro avente una notazione standard, 
un esempio di notazione standard sarebbe ARIMA con \(p, d\) e \(q\), dove valori interi sostituiscono i parametri
per indicare la tipologia del modello ARIMA utilizzato. I parametri posso essere definiti come:

\begin{itemize}
    \item \(p\): Il numero di osservazioni passate nel modello.
    \item \(d\): Il numero di volte che le osservazioni grezze sono state differenziate, chiamato anche grado di differenziazione.
    \item \(q\): la dimensione della finestra relativa alla media mobile, chiamato anche ordine della media mobile.
\end{itemize}