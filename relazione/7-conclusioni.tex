%%% CAPITOLO 7
%%% Conclusioni sul progetto

\section{Conclusioni}

In questo progetto di Business Intelligence sono stati visti visti numerosi argomenti utili per ottenere 
informazioni e prendere decisioni critiche, a partire da numerose statistiche descrittive che hanno
dimostrato importanti proprietà sui dati considerati come la elevata volatilità, predizione sul prezzo a partire
dai dati utilizzando varie metodologie tra cui modelli statistici e deep learning, costruzione di un algoritmo
per trading automatico e backtesting sulla strategia, stima sul rischio dei titoli considerati e in fine alla costruzione
di un portfolio ottimizzato utilizzando anche le predizioni effettuate precedentemente.

Relativamente alle analisi di previsione è stato reso noto come la alta volatilità dei titoli abbia influito sulla
previsione con modello \verb|ARIMA|, il titolo \verb|GOOG| è stato l'esempio più noto dove nel periodo di verifica il
prezzo reale si è discostato oltre l'intervallo di confidenza.\\
Considerando la elevata volatilità dei titoli l'utilizzo di un algoritmo di deep learning avrebbe potuto ottenere dati di previsione più precisi,
tuttavia dei test effettuati sul modello MLP hanno portato un livello di errore superiore a quello di naïve, segnalando la imprecisione del modello.

Nelle strategie di trading è stato molto interessante osservare come indicatori tecnici, come le Bande di Bollinger possano
essere impiegate in un algoritmo per segnalare automaticamente quando effettuare l'acquisto o la vendita di un titolo, nel caso presentato nel capitolo
4 per esempio, durante backtesting si è riuscito nel periodo del 2020 ad avere un profitto, tuttavia anche se non presentato nel capitolo ci sono
stati periodi come nel 2021 dove alla fine dell'anno il backtesting ha segnalato una perdita sul capitale investito inizialmente, ciò ha dimostrato che nonostante in
alcuni casi l'algoritmo presentato funzioni, può comunque portare a una perdita del capitale investito, e per questo motivo è necessario il backtesting per cercare di creare
algoritmi che mediante diverse strategie limitino il rischio di perdita del capitale.

Sui titoli considerati in questo progetto, nei capitoli 6 e 7 ne è stato calcolato il rischio utilizzando il \verb|CAPM| per valutare il rischio e successivamente sono stati costruiti due portafogli "ottimizzati"
utilizzando le simulazioni di monte carlo, il primo utilizzava unicamente i dati storici mentre il secondo usava sia i dati storici che quelli ottenuti tramite forecasting nel punto 3,
la costruzione dei due portafogli è stata utile per evidenziare la dispersione notevolmente inferiore quando sono stati utilizzati i dati di autoregressione provenienti da \verb|ARIMA|.\\
Per confrontare i portafogli ottimizzati sono stati costituiti successivamente altri due portafogli dove ogni titolo aveva lo stesso peso, sfruttando il concetto del \textbf{1/n portfolio}, in questo
caso con solo i dati passati è stato evidenziato un rendimento inferiore con una volatilità più alta, mentre con i dati di previsione il portfolio \verb|1/n| è riuscito ad avere una volatilità inferiore a discapito tuttavia
del rendimento nettamente più basso.