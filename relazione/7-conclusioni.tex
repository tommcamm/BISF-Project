%%% CAPITOLO 7
%%% Conclusioni sul progetto

\section{Conclusioni}

In questo progetto di Business Intelligence sono stati visti visti numerosi argomenti utili per ottenere 
informazioni e prendere decisioni critiche, a partire da numerose statistiche descrittive che hanno
dimostrato importanti proprietà sui dati considerati come la elevata volatilità, predizione sul prezzo a partire
dai dati utilizzando varie metodologie tra cui modelli statistici e deep learning, costruzione di un algoritmo
per trading automatico e backtesting sulla strategia, stima sul rischio dei titoli considerati e in fine alla costruzione
di un portfolio ottimizzato utilizzando anche le predizioni effettuate precedentemente.

Relativamente alle analisi di previsione è stato reso noto come la alta volatilità dei titoli abbia influito sulla
previsione con modello \verb|ARIMA|, il titolo \verb|GOOG| è stato l'esempio più noto dove nel periodo di verifica il
prezzo reale si è discostato oltre l'intervallo di confidenza.

Nelle strategie di trading è stato molto interessante osservare come indicatori tecnici, come le Bande di Bollinger possano
essere impiegate in un algoritmo per segnalare automaticamente quando comprare o vendere un titolo, nel caso presentato nel capitolo
4 per esempio, durante backtesting si è riuscito nel periodo del 2020 ad avere un profitto, tuttavia anche se non presentato ci sono
stati periodi come quello del 2021 dove alla fine dell'anno il backtesting ha segnalato una perdita, ciò ha dimostrato che nonostante In
alcuni casi l'algoritmo presentato funzioni, può comunque portare a una perdita del capitale investito.

