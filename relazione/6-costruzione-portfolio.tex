\section{Costruzione di un portfolio}

Per costruzione di un portfolio intendiamo distribuire la quota di investimento su certi asset (che siano stock, opzioni, bonds o qualsiasi altro strumento finanziario), l'obiettivo primario rimane nel bilanciare il rischio ed i potenziali profitti.

Il principio fondamentale utilizzato per la allocazione degli asset nel portfolio che stiamo creando di chiama \textbf{Modern Portfolio Theory} (\textbf{MPT} o analisi con media-varianza), MPT è stata creata
per aiutare gli investitori nella costruzione di un portfolio che massimizza i rendimenti per un livello di rischio specificato.

MPT è legato al concetto di \emph{diversificazione}, ciò significa che possedere diversi tipi di asset riduce il rischio, in quanto la perdita di rendimento di una particolare security ha meno impatto
sulla performance di portfolio. In principio minore è la correlazione tra gli asset nel portfolio, meglio è per la diversificazione.

\subsection{Costruzione del portfolio ottimale}

\subsubsection{metodo analitico}

Costruiamo ora un portfolio ottimale utilizzando un metodo analitico chiamato \textbf{1/$n$ portfolio} questo metodo è nato
in quanto è stato mostrato da vari matematici che è difficoltoso battere la performance di un $1/n$ portfolio utilizzando Strategie
di allocazione asset più avanzate.