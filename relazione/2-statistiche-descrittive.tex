%%% CAPITOLO 2
%%% STATISTICHE DESCRITTIVE

\section{Statistiche descrittive}

\subsection{Rendimenti semplici e composti}

Dato che il prezzo degli asset spesso non è \emph{stazionario} (media e varianza cambia nel tempo), ci viene comodo trasformare i prezzi in rendimenti per cercare
di rendere la serie temporale stazionaria (la proprietà desiderata per la modellazione statistica).

\subsubsection{Titoli Tecnologici}

Per i titoli GOOG e FB sono stati calcolati i rendimenti semplici netti (grafico a figura \ref{fig:rendimenti_semplici_tecno}) e i rendimenti compositi (grafico a figura \ref{fig:rendimenti_compositi_tecno}) e posti a confronto
si nota una generale correlazione nei rendimenti.

\begin{figure}[ht]
    \centering
    \begin{minipage}{.5\textwidth}
      \centering
      \includegraphics[width=1\linewidth]{tecno_rendimenti_semplici_netti.png}
      \captionof{figure}{Rendimenti semplici netti FB e GOOG}
      \label{fig:rendimenti_semplici_tecno}
    \end{minipage}%
    \begin{minipage}{.5\textwidth}
      \centering
      \includegraphics[width=.97\linewidth]{tecno_rendimenti_composti.png}
      \captionof{figure}{Rendimenti compositi FB e GOOG}
      \label{fig:rendimenti_compositi_tecno}
    \end{minipage}
\end{figure}

\textbf{Note sui titoli tecnologici}

Confrontando le serie storiche di GOOG e FB nel grafico a figura \ref{fig:all_stocks_price}, utilizzando la funzione \verb|.corr()| di pandas si mostra un forte correlazione positiva di \verb|0.962272| (figura \ref{fig:corr_tecno})
tra i due titoli, come dimostrato anche dai simili rendimenti compositi (ad eccezione per alcuni eventi).

\begin{figure}[ht]
  \centering
  \includegraphics[width=0.4\textwidth]{corr_tecno.png}
  \caption{tabella con correlazione titoli GOOG e FB (metodo di Pearson)}
  \label{fig:corr_tecno}
\end{figure}

\pagebreak

\textbf{Note sui rendimenti di Meta (FB)}

Osservando il grafico in figura \ref{fig:rendimenti_semplici_tecno} relativamente ai rendimenti semplici di FB, notiamo \textbf{3} eventi di notevole discostamento dalla media.

Nel primo trimestre del 2012 si nota un significativo crollo, analizzando notizie e articoli si può ricondurre il crollo a scetticismo che c`è stato
tra gli investitori\footnote{\href{https://money.cnn.com/2012/05/22/markets/facebook-shares/index.htm}{https://money.cnn.com/2012/05/22/markets/facebook-shares/index.htm}} in quanto
Facebook (ora Meta) all'epoca era stata appena quotata in borsa, il crollo è stato circa del 35\% solo nei primi mesi.

Nel terzo trimestre del 2013 tuttavia viene evidenziato una crescita sostanziale rispetto alla media, nonostante il crollo capitato poco dopo la quotazione in borsa nel 2012,
a metà anno il prezzo di FB è riuscito a raggiungere di nuovo il prezzo di \emph{IPO}\footnote{Initial Public Offering, \href{https://www.investopedia.com/terms/i/ipo.asp}{https://www.investopedia.com/terms/i/ipo.asp}}
raggiungendo poi verso il terzo trimestre la quota record del tempo di 50\$, in base a vari 
articoli\footnote{
  \href{https://www.reuters.com/article/us-facebook-ipoprice-idUSBRE96T1CI20130730}{https://www.reuters.com/article/us-facebook-ipoprice-idUSBRE96T1CI20130730},
  
  \hspace{1.5mm} \href{https://money.cnn.com/2013/09/26/investing/facebook-stock/index.html}{https://money.cnn.com/2013/09/26/investing/facebook-stock/index.html}
  }
si evidenzia come la crescita possa essere attribuita dalla introduzione delle pubblicità su dispositivi mobile (precedentemente la pubblicità era mostrata solo da sito web),
a luglio 2013 facebook ha infatti annunciato che la pubblicità su mobile ha contribuito al 41\% delle loro vendite nel secondo semestre.

Tra il primo ed il secondo trimestre del 2020, come evidenziato dal grafico c'è stato un grande discostamento dalla media, inizialmente molto negativo ma poco dopo molto positivo,
la caduta inziale del prezzo la si può attribuire alla crisi finanziaria del 2020, scatenata dalla epidemia da 
covid-19\footnote{
  \href{https://www.investopedia.com/facebook-stock-crashes-into-bear-market-territory-4800598}{https://www.investopedia.com/facebook-stock-crashes-into-bear-market-territory-4800598}
}.\\
Tuttavia grazie alla natura digitale del servizio, dunque non esposta alla epidemia come altre attività e all'annuncio di \emph{Facebook Shops} il prezzo ha velocemente raggiunto il valore pre-crollo arrivando
addirittura ad un record del tempo (20-05-2020) di 
\$230.75\footnote{
  \href{https://www.cnbc.com/2020/05/20/facebook-shares-reach-all-time-high-after-announcing-online-shopping-feature.html}{https://www.cnbc.com/2020/05/20/facebook-shares-reach-all-time-high-after-announcing-online-shopping-feature.html}
}\\

\textbf{Note sui rendimenti di Alphabet (GOOG)}

Sul titolo GOOG si notano distaccamenti negativi sul rendimento di entità notevolmente inferiore rispetto a FB, tuttavia si hanno distaccamenti positivi più numerosi (il net return raggiunge 0.1 più volte rispetto a FB) anche se spesso non oltre 0.1.

Come per FB, ad aprile 2020 c'è stato un decremento significativo del prezzo (che si è riflettuto sul rendimento) a causa della epidemia da 
covid-19\footnote{
  \href{https://www.investopedia.com/alphabet-stock-crashes-into-bear-market-territory-4800600}{https://www.investopedia.com/alphabet-stock-crashes-into-bear-market-territory-4800600}
}
, ma sempre per il fatto che Alphabet Inc. fornisce
prevalentemente servizi digitali il prezzo è ritornato al valore pre-crollo velocemente.

\subsubsection{Titoli Militari}

Per i titoli RTX e LMT sono stati calcolati i rendimenti semplici netti (grafico a figura \ref{fig:rendimenti_semplici_mil}) e i rendimenti compositi (grafico a figura \ref{fig:rendimenti_compositi_mil}) e posti a confronto
si nota una generale correlazione nei rendimenti, anche se di entità inferiore rispetto a FB/GOOG.

\begin{figure}[ht]
  \centering
  \begin{minipage}{.5\textwidth}
    \centering
    \includegraphics[width=1\linewidth]{mil_rendimenti_semplici_netti.png}
    \captionof{figure}{Rendimenti semplici netti RTX e LMT}
    \label{fig:rendimenti_semplici_mil}
  \end{minipage}%
  \begin{minipage}{.5\textwidth}
    \centering
    \includegraphics[width=.97\linewidth]{mil_rendimenti_composti.png}
    \captionof{figure}{Rendimenti compositi RTX e LMT}
    \label{fig:rendimenti_compositi_mil}
  \end{minipage}
\end{figure}

Confrontando i titoli RTX e LMT usando sempre il grafico a figura \ref{fig:all_stocks_price} e utilizzando la funzione \verb|.corr()| di pandas
viene mostrata anche in questo caso una \textbf{forte correlazione positiva} di \verb|0.836831| (figura \ref{fig:corr_mill}).

\begin{figure}[ht]
  \centering
  \includegraphics[width=0.4\textwidth]{corr_mil.png}
  \caption{tabella con correlazione titoli RTX e LMT (metodo di Pearson)}
  \label{fig:corr_mill}
\end{figure}

\textbf{Note sui rendimenti di Raytheon (RTX)}

Osservando il grafico in figura \ref{fig:rendimenti_semplici_mil} relativamente ai rendimenti semplici di RTX, si nota come il nel tempo è
stato molto fluido, troviamo \textbf{3} eventi notevoli.

Sul titolo RTX è stato più complicato la ricerca di notizie coerenti con il crollo del prezzo in quanto è notevolmente meno famoso rispetto a FB o GOOG,
tuttavia relativamente al primo ed al secondo crollo si può assumere siano in relazione alla politica di export delle armi americane.

Il crollo più notevole è accaduto durante la crisi finanziaria del 2020, in base a un 
articolo\footnote{
  \href{https://www.fool.com/investing/2020/07/02/heres-why-raytheon-technologies-shares-are-down-34.aspx}{https://www.fool.com/investing/2020/07/02/heres-why-raytheon-technologies-shares-are-down-34.aspx}
}
la motivazione del crollo è stata a causa della temporanea debolezza nel settore commerciale aerospaziale, causata dalla epidemia da COVID-19,
si assume comunque che essendo una compagnia incentrata nel settore aerospaziale sarà comunque una migliore scelta rispetto alla competizione per il
settore della difesa.\\

\textbf{Note sui rendimenti di Lockheed Martin (LMT)}

Il titolo LMT ha subito variazioni molto meno significanti rispetto a RTX e gli altri titoli (come si può vedere dal grafico a figura \ref{fig:rendimenti_semplici_mil}), analizzando varie notizie nel web si identificano \textbf{2} eventi significativi.

Tra fine 2018 e l'inizio del 2019 si è registrato un crollo del 18.4\% relativo allo stock LMT, in base ad un 
articolo\footnote{
  \href{https://www.fool.com/investing/2019/01/11/heres-why-lockheed-martin-lost-184-in-2018.aspx}{https://www.fool.com/investing/2019/01/11/heres-why-lockheed-martin-lost-184-in-2018.aspx}
}
le cause del crollo sono molteplici, uno delle cause identificate è una continua diminuzione del budget dalla casa bianca, una altra è la dimissione improvvisa del 
CFO\footnote{
  Chief Financial Officer, \href{https://en.wikipedia.org/wiki/Chief_financial_officer}{https://en.wikipedia.org/wiki/Chief\_financial\_officer}
}
in quanto era ben visto dagli investitori per le sue abilità comunicative.

Il crollo più importante del prezzo è stato durante la crisi finanziaria del 2020, dopo il crollo nel primo trimestre tuttavia c'è stato un notevole recupero poco dopo messo poi a rischio verso ottobre a causa di problemi nella supply chain della
produzione\footnote{
  \href{https://www.investopedia.com/lockheed-martin-lmt-sells-off-despite-strong-quarter-5083204}{https://www.investopedia.com/lockheed-martin-lmt-sells-off-despite-strong-quarter-5083204}
}
causato dal tilt nelle fabbriche e nei trasporti.

Nello stesso articolo si evidenzia inoltre come la motivazione della instabilità verso la fine del 2020 sia anche causata da una incertezza politica, in quanto si assumeva che una vittoria democratica avrebbe tagliato il
budget alla difesa, anche se il rischio di guerra nucleare dovrebbe evitare un crollo del titolo.

\subsubsection{Titoli Bancari}

Per i titoli BAC e JPM sono stati calcolati i rendimenti semplici netti (grafico a figura \ref{fig:rendimenti_semplici_banc}) e i rendimenti composti (grafico a figura \ref{fig:rendimenti_compositi_banc}) e posti a confronto
anche per questi due titoli si nota una forte correlazione nei rendimenti.

\begin{figure}[ht]
  \centering
  \begin{minipage}{.5\textwidth}
    \centering
    \includegraphics[width=1\linewidth]{banc_rendimenti_semplici_netti.png}
    \captionof{figure}{Rendimenti semplici netti BAC e JPM}
    \label{fig:rendimenti_semplici_banc}
  \end{minipage}%
  \begin{minipage}{.5\textwidth}
    \centering
    \includegraphics[width=.97\linewidth]{banc_rendimenti_composti.png}
    \captionof{figure}{Rendimenti compositi BAC e JPM}
    \label{fig:rendimenti_compositi_banc}
  \end{minipage}
\end{figure}

Confrontando i titoli BAC e JPM usando il grafico a figura \ref{fig:all_stocks_price} e utilizzando la funzione \verb|.corr()| di pandas
viene mostrata una \textbf{forte correlazione positiva} di \verb|0.909522| (figura \ref{fig:corr_banc}).

\begin{figure}[ht]
  \centering
  \includegraphics[width=0.4\textwidth]{corr_banc.png}
  \caption{tabella con correlazione titoli BAC e JPM (metodo di Pearson)}
  \label{fig:corr_banc}
\end{figure}

\textbf{Note sui rendimenti di Bank of America (BAC)}

Il titolo BAC ha in multiple occasioni dei sostanziali aumenti di prezzo ma anche dei crolli, identifichiamo almeno \textbf{5} eventi notevoli.

Nel primo trimestre del 2012 si nota un netto crollo del prezzo, facendo delle ricerche ho trovato un 
articolo\footnote{
  \href{https://www.reuters.com/article/uk-bofa-lawsuit-idUKBRE88R11X20120928}{https://www.reuters.com/article/uk-bofa-lawsuit-idUKBRE88R11X20120928}
}
che spiega come BAC ha accettato di pagare 2.3 milioni di dollari agli investitori per patteggiare una causa legale per gli eventi legati alla crisi finanziaria del 2008.
Essendo un periodo vicino al 2008 si assume anche che la alta volatilità si anche data dalla poca fiducia degli investitori in questo titolo.

Nel 2016 si registra un altro periodo di instabilità per questo titolo, risultato poi verso la fine dell'anno con un aumento del 34\% sul prezzo,
secondo questo 
articolo\footnote{
  \href{https://www.fool.com/investing/2016/12/29/why-bank-of-americas-stock-climbed-34-in-2016.aspx}{https://www.fool.com/investing/2016/12/29/why-bank-of-americas-stock-climbed-34-in-2016.aspx}
}
il declino del prezzo all'inizio dell'anno è stato a causa del crollo dei prezzi relativi all'energia,
successivamente i prezzi del settore energetico ci hanno messo poco tempo a recuperare, tuttavia nonostante il recupero a metà anno il prezzo di BAC è comunque sceso,
la causa della seconda declinazione dei prezzi, si può attribuire alla improvvisa decisione dell'inghilterra di uscire dalla unione europea (brexit), che ha causato un momento
di incertezza finanziaria, sopratutto con Bank of America in quanto a causa di brexit avrebbe dovuto spostare parte delle proprie operazioni in un altro paese europeo.\\
Verso la fine dell'anno poi c'è stato una crescita notevole e inaspettata del prezzo, secondo l'articolo questa crescita può essere stata a causa della vittoria del partito
Repubblicano delle elezioni presidenziali, in quanto il candidato presidente aveva promesso la rimozione di regolatorie instaurate nel 2010 dal precedente Presidente.

Nel 2020 come per i titoli mostrati in precedenza c'è stato un grosso crollo del prezzo di BAC a causa della crisi finanziaria del 2020, secondo questo 
articolo\footnote{
  \href{https://www.investopedia.com/bank-of-america-earnings-4770948}{https://www.investopedia.com/bank-of-america-earnings-4770948}
}
nonostante l'ottimismo del CEO riguardante il futuro recupero dal crollo, i mesi successivi sono rimasti altamente instabili e non hanno portato come per i titoli tecnologici
a un aumento stabile.\\

\textbf{Note sui rendimenti di JPMorgan Chase (JPM)}

Controllando i grafici relativi al prezzo e rendimenti, si può notare come il titolo JPM sia stato molto correlato con BAC, in quanto operano nello stesso settore e sono entrambi
titoli finanziari.\\
Relativamente agli eventi del 2012 e 2016 si può assumere che gli stessi eventi di BAC abbiano causato il crollo sui rendimenti e quindi periodi di instabilità.

Relativamente alla crisi finanziaria del 2020, secondo questo
articolo\footnote{
  \href{https://www.cnbc.com/2020/04/14/jpmorgan-chase-jpm-earnings-q1-2020.html}{https://www.cnbc.com/2020/04/14/jpmorgan-chase-jpm-earnings-q1-2020.html}
}
oltre al crollo del prezzo a causa della epidemia, c'è stato un enorme movimento di 6.8 milioni di dollari alla riserva di credito della banca, movimento che si assume sia stato effettuato
per proteggere la banca da un aumento di default relativi alle attività che hanno effettuato prestiti con JPM, questo movimento ha causato un ulteriore crollo del prezzo che si è propagato nei mesi successivi.

\pagebreak

\subsection{Istogrammi sui rendimenti e dispersione}

Per misurare la dispersione sui titoli ci torna utile il concetto di deviazione standard, grazie ad essa si può avere una idea della
 volatilità associata al titolo, un dato utile per effettuare investimenti e/o strategie di 
trading\footnote{
  \href{https://www.investopedia.com/terms/s/standarddeviation.asp}{https://www.investopedia.com/terms/s/standarddeviation.asp}
}.\\

\subsubsection{Titoli tecnologici (Meta/Alphabet)}

L'istogramma del rendimento relativo ai titoli GOOG e FB può essere visto a figura \ref{fig:isto_rendimenti_tecno}, mentre la distribuzione dei 
rendimenti si trova nel grafico in figura \ref{fig:dispersione_tecno}.
Dai dati si evidenzia come la maggior parte dei rendimenti avviene tra \verb|0.0| e \verb|0.1|, oltre al fatto che alphabet nel periodo di interesse ha avuto rendimenti più alti
rispetto a Meta.

\begin{figure}[ht]
  \centering
  \begin{minipage}{.5\textwidth}
    \centering
    \includegraphics[width=1\linewidth]{net_ret_tecno_hist.png}
    \captionof{figure}{Istogramma rendimenti FB e GOOG}
    \label{fig:isto_rendimenti_tecno}
  \end{minipage}%
  \begin{minipage}{.5\textwidth}
    \centering
    \includegraphics[width=1\linewidth]{dispersione_tecno.png}
    \captionof{figure}{Distribuzione di FB e GOOG}
    \label{fig:dispersione_tecno}
  \end{minipage}
\end{figure}

Inoltre é stata calcolata grazie a \emph{Pandas} la Deviazione Standard, dove il risultato si può vedere nella tabella a figura \ref{fig:ds_tecno}.

\begin{figure}[ht]
  \centering
  \includegraphics[width=0.4\textwidth]{ds_tecno.png}
  \caption{Deviazione Standard dei titoli FB e GOOG}
  \label{fig:ds_tecno}
\end{figure}

\subsubsection{Titoli militari (Raytheon/Lockheed Martin)}

L'istogramma del rendimento relativo ai titoli RTX e LMT può essere visto nel grafico a figura \ref{fig:isto_rendimenti_mil}, mentre la distribuzione dei rendimenti
si trova nel grafico a figura \ref{fig:dispersione_mil}.\\
Dai dati si evidenzia che i rendimenti si trovano per la maggior parte tra \verb|-0.05| e \verb|0.5|, Lockheed martin ha avuto rendimenti più alti rispetto
a Raytheon.

\begin{figure}[ht]
  \centering
  \begin{minipage}{.5\textwidth}
    \centering
    \includegraphics[width=1\linewidth]{net_ret_mil_hist.png}
    \captionof{figure}{Istogramma rendimenti RTX e LMT}
    \label{fig:isto_rendimenti_mil}
  \end{minipage}%
  \begin{minipage}{.5\textwidth}
    \centering
    \includegraphics[width=1\linewidth]{dispersione_mil.png}
    \captionof{figure}{Distribuzione di RTX e LMT}
    \label{fig:dispersione_mil}
  \end{minipage}
\end{figure}

Sempre con \emph{pandas} è stata calcolata la Deviazione Standard, dove il risultato è riportato nella tabella a figura \ref{fig:ds_mil}.

\begin{figure}[ht]
  \centering
  \includegraphics[width=0.4\textwidth]{ds_mil.png}
  \caption{Deviazione Standard dei titoli RTX e LMT}
  \label{fig:ds_mil}
\end{figure}

\pagebreak

\subsubsection{Titoli bancari (Bank of America/JPMorgan Chase)}

L'istogramma per il rendimento dei titoli finanziari BAC e JPM si può vedere nel grafico nella figura \ref{fig:isto_rendimenti_banc}, mentre la distribuzione dei rendimenti
si può vedere nel grafico in figura \ref{fig:dispersione_banc}.\\
Dai dati si evidenzia che i rendimenti si trovano per la maggior parte tra \verb|-0.01| e \verb|0.1|, Bank of America ha avuto rendimenti più alti
rispetto a JPMorgan Chase.

\begin{figure}[ht]
  \centering
  \begin{minipage}{.5\textwidth}
    \centering
    \includegraphics[width=1\linewidth]{net_ret_banc_hist.png}
    \captionof{figure}{Istogramma rendimenti BAC e JPM}
    \label{fig:isto_rendimenti_banc}
  \end{minipage}%
  \begin{minipage}{.5\textwidth}
    \centering
    \includegraphics[width=1\linewidth]{dispersione_banc.png}
    \captionof{figure}{Distribuzione di BAC e JPM}
    \label{fig:dispersione_banc}
  \end{minipage}
\end{figure}

Con \emph{pandas} è stata calcolata la Deviazione Standard, dove il risultato è nella tabella a figura \ref{fig:ds_banc}.

\begin{figure}[ht]
  \centering
  \includegraphics[width=0.4\textwidth]{ds_banc.png}
  \caption{Deviazione Standard dei titoli BAC e JPM}
  \label{fig:ds_banc}
\end{figure}

\pagebreak

\subsection{Grafici diagnositici a 4 sezioni}

Vengono mostrati per i titoli considerati la serie di 4 grafici diagnostici (Istogramma, kernel density, Q-Q plot e boxplot).\\
Questi grafici rappresentano 4 modi diversi per rappresentare la \textbf{Distribuzione Normale}, fondamentale per studiare la distribuzione sui rendimenti dei titoli.

Questi grafici ci permettono inoltre di identificare gli \textbf{outliers}, osservazioni che sono significativamente differenti dalla maggioranza, è fondamentale prima di lavorare con algoritmi di machine learning identificare tali outliers in quanto possono influenzare i risultati
portando a risultati incorretti o con bias.

\subsubsection{Grafici Diagnostici per Meta Platforms, Inc. (FB)}

Per Meta, possiamo trovare i grafici alla figura \ref{fig:meta_diagn}, si può notare come i rendimenti sono distribuiti normalmente e simmetricamente,
Si notano inoltre degli outliers: due tra \verb|0.2| e \verb|0.35| e due tra \verb|-0.2| e \verb|-0.4|.

\begin{figure}[ht]
  \centering
  \includegraphics[width=0.7\textwidth]{meta_diagn.png}
  \caption{Grafici diagnostici per Meta (FB)}
  \label{fig:meta_diagn}
\end{figure}

\pagebreak

\subsubsection{Grafici Diagnostici per Alphabet Inc. (GOOG)}

Per Alphabet, possiamo trovare i grafici alla figura \ref{fig:goog_diagn}, possiamo notare anche qui che i rendimenti sono distribuiti normalmente, con una notevole inclinazione positiva rispetto a FB,
Anche per questo titolo si nota un outliers: tra \verb|-0.20| e \verb|-0.25|.

\vspace{3cm}

\begin{figure}[ht]
  \centering
  \includegraphics[width=0.7\textwidth]{goog_tecn.png}
  \caption{Grafici diagnostici per Alphabet (GOOG)}
  \label{fig:goog_diagn}
\end{figure}

\pagebreak

\subsubsection{Grafici Diagnostici per Raytheon Technologies Corporation. (RTX)}

Per Raytheon, i grafici si trovano alla figura \ref{fig:rtx_diagn}, possiamo notare che qui i rendimenti sono distribuiti normalmente e simmetricamente.
Per questo titolo troviamo un outliers: tra \verb|-0.35| e \verb|-0.4|.

\vspace{3cm}

\begin{figure}[ht]
  \centering
  \includegraphics[width=0.7\textwidth]{rtx_diagn.png}
  \caption{Grafici diagnostici per Raytheon (RTX)}
  \label{fig:rtx_diagn}
\end{figure}

\pagebreak

\subsubsection{Grafici Diagnostici per Lockheed Martin Corporation. (LMT)}

Per Lockheed Martin, i grafici si trovano alla figura \ref{fig:lmt_diagn}, anche qui i rendimenti sono distribuiti normalmente e in maniera simmetrica (la curva è più translata verso destra).
Per questo titoli abbiamo tre outliers: due tra \verb|-0.05| e \verb|-0.10|, e uno tra \verb|-0.20| e \verb|-0.25|.

\vspace{3cm}

\begin{figure}[ht]
  \centering
  \includegraphics[width=0.7\textwidth]{lmt_diagn.png}
  \caption{Grafici diagnostici per Lockheed Martin (LMT)}
  \label{fig:lmt_diagn}
\end{figure}

\pagebreak

\subsubsection{Grafici Diagnostici per Bank of America. (BAC)}

Per Bank of America, i grafici diagnostici sono alla figura \ref{fig:bac_diagn}, anche qui sono distribuiti generalmente normalmente e simmetricamente.
Per BAC abbiamo numerosi outliers: tre tra \verb|0.15| e \verb|0.25|, quattro o più tra \verb|-0.1| e \verb|-0.2| e uno tra \verb|-0.35| e \verb|-0.4|.

\vspace{3cm}

\begin{figure}[ht]
  \centering
  \includegraphics[width=0.7\textwidth]{bac_diagn.png}
  \caption{Grafici diagnostici per Bank of America (BAC)}
  \label{fig:bac_diagn}
\end{figure}

\pagebreak

\subsubsection{Grafici Diagnostici per JPMorgan Chase. (JPM)}

Per JPMorgan Chase, troviamo i grafici diagnostici alla figura \ref{fig:jpm_diagn}, la distribuzione è anche qui normale e leggermente inclinata verso destra.
Per questo titolo abbiamo 3 outliers: due tra \verb|-0.1| e \verb|-0.2| e uno tra \verb|-0.3| e \verb|-0.35|.

\vspace{3cm}

\begin{figure}[ht]
  \centering
  \includegraphics[width=0.7\textwidth]{jpm_diagn.png}
  \caption{Grafici diagnostici per JPMorgan Chase (JPM)}
  \label{fig:jpm_diagn}
\end{figure}

\pagebreak

\subsection{Statistiche descrittive univariate}

Per ogni serie di rendimenti sono state considerate le seguenti statistiche univariate: media, varianza, deviazione standard, asimmetria e curtosi.

Tali statistiche univariate servono per [TODO]

\textbf{Statistiche per Meta (FB)}

Per Meta identifichiamo una volatilità del \verb|36.44|\%, i grafici e la tabella sono a figura \ref{fig:fb_stats} e \ref{fig:fb_vol}.

\begin{figure}[ht]
  \centering
  \begin{minipage}{.5\textwidth}
    \centering
    \vspace{4.35cm}
    \includegraphics[width=1\linewidth]{meta_stats.png}
    \captionof{figure}{Statistiche univariate per Meta (FB)}
    \label{fig:fb_stats}
  \end{minipage}%
  \begin{minipage}{.5\textwidth}
    \centering
    \includegraphics[width=1\linewidth]{meta_volatility.png}
    \captionof{figure}{Grafico volatilità di Meta (FB)}
    \label{fig:fb_vol}
  \end{minipage}
\end{figure}

\textbf{Statistiche per Alphabet (GOOG)}

Per Alphabet identifichiamo una volatilità del \verb|25.11|\%, i grafici e la tabella sono a figura \ref{fig:goog_stats} e \ref{fig:goog_vol}.

\begin{figure}[ht]
  \centering
  \begin{minipage}{.5\textwidth}
    \centering
    \vspace{4.85cm}
    \includegraphics[width=1\linewidth]{goog_stats.png}
    \captionof{figure}{Statistiche univariate per Alphabet\\ (GOOG)}
    \label{fig:goog_stats}
  \end{minipage}%
  \begin{minipage}{.5\textwidth}
    \centering
    \includegraphics[width=1\linewidth]{goog_volatility.png}
    \captionof{figure}{Grafico volatilità di Alphabet (GOOG)}
    \label{fig:goog_vol}
  \end{minipage}
\end{figure}

\pagebreak

\textbf{Statistiche per Raytheon (RTX)}

Per Raytheon identifichiamo una volatilità del \verb|25.49|\%, i grafici e la tabella sono a figura \ref{fig:rtx_stats} e \ref{fig:rtx_vol}.

\begin{figure}[ht]
  \centering
  \begin{minipage}{.5\textwidth}
    \centering
    \vspace{4.85cm}
    \includegraphics[width=1\linewidth]{rtx_stats.png}
    \captionof{figure}{Statistiche univariate per Raytheon\\ (RTX)}
    \label{fig:rtx_stats}
  \end{minipage}%
  \begin{minipage}{.5\textwidth}
    \centering
    \includegraphics[width=1\linewidth]{rtx_volatility.png}
    \captionof{figure}{Grafico volatilità di Raytheon (RTX)}
    \label{fig:rtx_vol}
  \end{minipage}
\end{figure}

\textbf{Statistiche per Lockheed Martin (LMT)}

Per Lockheed Martin identifichiamo una volatilità del \verb|21.1|\%, i grafici e la tabella sono a figura \ref{fig:lmt_stats} e \ref{fig:lmt_vol}.

\begin{figure}[ht]
  \centering
  \begin{minipage}{.5\textwidth}
    \centering
    \vspace{4.35cm}
    \includegraphics[width=1\linewidth]{lmt_stats.png}
    \captionof{figure}{Statistiche univariate per\\ Lockheed Martin (LMT)}
    \label{fig:lmt_stats}
  \end{minipage}%
  \begin{minipage}{.5\textwidth}
    \centering
    \includegraphics[width=1\linewidth]{lmt_volatility.png}
    \captionof{figure}{Grafico volatilità di Lockheed Martin (LMT)}
    \label{fig:lmt_vol}
  \end{minipage}
\end{figure}

\pagebreak

\textbf{Statistiche per Bank of America (BAC)}

Per Bank of America identifichiamo una volatilità del \verb|31.74|\%, i grafici e la tabella sono a figura \ref{fig:bac_stats} e \ref{fig:bac_vol}.

\begin{figure}[ht]
  \centering
  \begin{minipage}{.5\textwidth}
    \centering
    \vspace{4.35cm}
    \includegraphics[width=1\linewidth]{bac_stats.png}
    \captionof{figure}{Statistiche univariate per\\ Bank of America (BAC)}
    \label{fig:bac_stats}
  \end{minipage}%
  \begin{minipage}{.5\textwidth}
    \centering
    \includegraphics[width=1\linewidth]{bac_volatility.png}
    \captionof{figure}{Grafico volatilità di Bank of America (BAC)}
    \label{fig:bac_vol}
  \end{minipage}
\end{figure}

\textbf{Statistiche per JPMorgan Chase (JPM)}

Per JPMorgan  Chase identifichiamo una volatilità del \verb|27.04|\%, i grafici e la tabella sono a figura \ref{fig:jpm_stats} e \ref{fig:jpm_vol}.

\begin{figure}[ht]
  \centering
  \begin{minipage}{.5\textwidth}
    \centering
    \vspace{4.35cm}
    \includegraphics[width=1\linewidth]{jpm_stats.png}
    \captionof{figure}{Statistiche univariate per\\ JPMorgan Chase (JPM)}
    \label{fig:jpm_stats}
  \end{minipage}%
  \begin{minipage}{.5\textwidth}
    \centering
    \includegraphics[width=1\linewidth]{jpm_volatility.png}
    \captionof{figure}{Grafico volatilità di JPMorgan Chase (JPM)}
    \label{fig:jpm_vol}
  \end{minipage}
\end{figure}

\textbf{Note sulle statistiche presentate qui sopra}

Basandosi sulle statistiche descrittive univariate presentate qui sopra possiamo concludere che:

\begin{itemize}
  \item Utilizzando la \textbf{media} si è notato che, \verb|Meta| (FB) e \verb|Bank of America| (BAC) hanno il rendimento più alto,
  mentre \verb|Raytheon| (RTX) e \verb|Lockheed Martin| (LMT) hanno il rendimento più basso.
  \item Il titolo \verb|Meta| (FB) ha una distribuzione dei rendimenti più vicina alla distribuzione normale, mentre 
  \verb|Raytheon| (RTX) è più lontano dalla normale.
  \item La volatilità più alta è stata riscontrata tra le azioni di \verb|Meta| (FB) e \verb|Bank of America| (BAC), invece
  la più bassa tra \verb|Lockheed Martin| (LMT) e \verb|Alphabet| (GOOG).
  \item La deviazione standard più bassa (indice di rischio per un asset) la hanno \verb|Lockheed Martin| (LMT) e \verb|Alphabet| (GOOG), mentre
  quella più alta la hanno \verb|Meta| (FB) e \verb|Bank of America| (BAC).
\end{itemize}

\pagebreak

%% == SEZ 2.5 ==
%% Matrice di varianze/covarianze dei rendimenti
%% data: https://support.minitab.com/en-us/minitab/18/help-and-how-to/modeling-statistics/anova/supporting-topics/anova-statistics/what-is-the-variance-covariance-matrix/

\subsection{Matrice di varianze/covarianze dei rendimenti}

La matrice di varianze/covarianze è una particolare matrice dove sulla diagonale abbiamo la varianza del relativo titoli e nelle altre posizioni abbiamo il valore di covarianza
della coppia di titoli, questa matrice ci permette quindi di osservare questi due valori in una sola matrice.

\textbf{Covarianza:} La covarianza \(cov\) è una misura che rappresenta come si muove la media di due variabili aleatorie. Nella finanza è utilizzata per determinare la relazione direzionale
tra i rendimenti di due asset, una covarianza positiva indica che i rendimenti dei due asset si muovono insieme mentre una covarianza negativa indica che si muovono inversamente.

\begin{displaymath}
  cov(x,y) = \frac{(x_i - x_m) \cdot (y_i - y_m)}{n - 1}
\end{displaymath}

\textbf{dove}\\
\(x_i = \text{Rendimento i-esimo del titolo X}\)\\
\(x_m = \text{La media del titolo X del dataset}\)\\
\(y_i = \text{Rendimento i-esimo del titolo Y}\)\\
\(y_m = \text{La media del titolo Y del dataset}\)

\textbf{Varianza:} La varianza \(\sigma^2\) è una misura statistica utilizzata per determinare la distanza dei numeri dalla loro media in un dataset, in finanza è utilizzata
per determinare la volatilità di un asset.

\begin{displaymath}
  \sigma^2 = \frac{\sum(x_i - \overline{x})^2}{n - 1}  = cov(x,x)
\end{displaymath}

\begin{displaymath}
  mat_{\text{var/cov}} = 
\begin{bmatrix}
  \sigma^2_1 & cov(1,2) & \cdots & cov(1,j)\\
  cov(2,1) & \sigma^2_2 & \cdots & cov(2,j)\\
  \vdots   & \vdots   & \ddots & \vdots\\
  cov(i,1) & cov(i,2) & \cdots & \sigma^2_i
\end{bmatrix}
\end{displaymath}

\begin{figure}[ht]
  \centering
  \includegraphics[width=0.6\textwidth]{var_cov_matrix.png}
  \caption{Matrice di Varianza/Covarianza su tutti i titoli}
  \label{fig:var_cov_matrix}
\end{figure}

Considerando la tabella in figura \ref{fig:var_cov_matrix}, si può notare come \verb|JPM| e \verb|BAC| hanno un valore di Covarianza
relativamente alto del \verb|2.88|\%. Tale valore indica uno stesso comportamento in caso di aumento del prezzo o crollo, questa considerazione implica
un maggior rischio nel nostro portfolio.

Si nota inoltre che \verb|GOOG| e \verb|LMT| hanno una covarianza del \verb|0.77|\%, questo valore più vicino allo zero rispetto alle altre coppie implica
un minor rischio nel portfolio tra questi due titoli.

%% == SEZ 2.6 ==
%% Matrice di correlazione dei rendimenti 
%% data: https://www.investopedia.com/terms/c/correlation.asp

\subsection{Matrice di correlazione dei rendimenti}

La matrice di correlazione dei rendimenti ci permette di osservare per ogni coppia di titoli il valore di correlazione sui rendimenti.

\textbf{Correlazione:} La correlazione \(r\) è un termine statistico che descrive il grado di coordinazione tra due variabili, se le due variabili si muovono nella stessa direzione
la coordinazione è positiva, se si muovono in direzione opposta allora è negativa.\\
Il valore massimo è \verb|1.0| mentre quello minimo è \verb|-1.0|, quando ci avviciniamo agli estremi, si dice che la correlazione è \textbf{forte}, altrimenti se il valore è vicino allo zero si parla di correlazione \textbf{bassa} (o non esistente).

\begin{displaymath}
  r = \frac{n \cdot (\sum (X, Y) - (\sum (X) \cdot \sum (Y)))}
  {\sqrt{(n \cdot \sum(X^2) - \sum (X)^2) \cdot (n \cdot \sum (Y^2) - \sum (Y)^2)}}
\end{displaymath}

\textbf{dove}\\
\(r = \text{Coefficiente di correlazione}\)\\
\(n = \text{Numero di osservazioni}\)

Nel caso dei titoli finanziari la correlazione è un valore importante in quanto ci permette di effettuare predizioni sui trend futuri e quindi gestire il rischio del nostro
portfolio.

\begin{figure}[ht]
  \centering
  \includegraphics[width=0.6\textwidth]{corr_matrix.png}
  \caption{Matrice di correlazione sui rendimenti}
  \label{fig:corr_matrix}
\end{figure}

Dalla matrice di correlazione a figura \ref{fig:corr_matrix} notiamo che i titoli più correlati sono \verb|JPM| e \verb|BAC| con un valore di correlazione pari a
\verb|0.88|. I titoli meno correlati sono invece \verb|FB| e \verb|LMT| con un valore di correlazione pari a \verb|0.251|.

\subsection{Grafico sulla correlazione dei titoli nel tempo e scatter plot}

\subsubsection{Titoli tecnologici FB/GOOG}

\begin{figure}[ht]
  \centering
  \begin{minipage}{.4\textwidth}
    \centering
    \includegraphics[width=.5\linewidth]{fb_goog_log_return.png}
    \captionof{figure}{Ritorni logaritmici mensili di FB/GOOG per l'anno 2020}
    \label{fig:fb_goog_log_return}
  \end{minipage}%
  \begin{minipage}{.6\textwidth}
    \centering
    \includegraphics[width=1\linewidth]{fb_goog_corr.png}
    \captionof{figure}{Correlazione nel tempo di FB/GOOG}
    \label{fig:fb_goog_corr}
  \end{minipage}
\end{figure}

Dal grafico in figura \ref{fig:fb_goog_corr} si evince che c'è sempre stata tra i due titoli una correlazione positiva, dopo
il 2015 viene mostrato come la correlazione supera \verb|0.5| e per tutta la durata successiva rimane oltre questo valore.
La tabella a figura \ref{fig:fb_goog_log_return} la forte correlazione in quanto in tutto l'anno i rendimenti hanno avuto lo stesso andamento.

\begin{figure}[ht]
  \centering
  \includegraphics[width=0.6\textwidth]{fb_goog_scatter.png}
  \caption{Dispersione rendimenti FB/GOOG}
  \label{fig:fb_goog_disp}
\end{figure}

Lo scatter plot a figura \ref{fig:fb_goog_corr} evidenza e conferma la forte correlazione tra i due titoli (seppur ci sono alcuni punti lontani dal centro).

\pagebreak

\subsubsection{Titoli militari RTX/LMT}

\begin{figure}[ht]
  \centering
  \begin{minipage}{.4\textwidth}
    \centering
    \includegraphics[width=.5\linewidth]{rtx_lmt_log_return_2017.png}
    \captionof{figure}{Ritorni logaritmici mensili di RTX/LMT per l'anno 2017}
    \label{fig:rtx_lmt_log_return_2017}
  \end{minipage}%
  \begin{minipage}{.6\textwidth}
    \centering
    \includegraphics[width=1\linewidth]{rtx_lmt_corr.png}
    \captionof{figure}{Correlazione nel tempo di RTX/LMT}
    \label{fig:rtx_lmt_corr}
  \end{minipage}
\end{figure}

Dal grafico in figura \ref{fig:rtx_lmt_corr} viene mostrato come ci sono stati nel tempo momenti di correlazione alta, e momenti dove c'è stato un crollo nella correlazione, nel dettaglio
fino al 2016 la correlazione è rimasta stabile tra \verb|0.55| e \verb|0.65|, dal 2016 c'è stato un notevole crollo che ha permesso raggiungere  valori di correlazione inferiori a \verb|0.4|,
verso il 2018 la correlazione è ritornata ai valori prima del crollo, con addirittura uno spike poco dopo il 2020, presumibilmente a causa delle crisi finanziaria del 2020.

Nella tabella a figura \ref{fig:rtx_lmt_log_return_2017} vengono mostrati i rendimenti logaritmici per l'anno 2017, si può notare come siano
nettamente meno simili i rendimenti tra i due titoli, a figura \ref{fig:rtx_lmt_log_return_2020} viene mostrato invece i rendimenti per l'anno 2020, in questo caso si nota come i rendimenti sono più "vicini".

\begin{figure}[ht]
  \centering
  \begin{minipage}{.4\textwidth}
    \centering
    \includegraphics[width=.5\linewidth]{rtx_lmt_log_return_2020.png}
    \captionof{figure}{Ritorni logaritmici mensili di RTX/LMT per l'anno 2020}
    \label{fig:rtx_lmt_log_return_2020}
  \end{minipage}%
  \begin{minipage}{.6\textwidth}
    \centering
    \includegraphics[width=.83\linewidth]{rtx_lmt_disp.png}
    \captionof{figure}{Dispersione rendimenti RTX/LMT}
    \label{fig:rtx_lmt_disp}
  \end{minipage}
\end{figure}

Lo scatter plot a figura \ref{fig:rtx_lmt_disp} dimostra come la correlazione sia più debole rispetto ai titoli precedenti.

\pagebreak

\subsubsection{Titoli bancari BAC/JPM}

\begin{figure}[ht]
  \centering
  \begin{minipage}{.4\textwidth}
    \centering
    \includegraphics[width=.5\linewidth]{bac_jpm_log_return_2020.png}
    \captionof{figure}{Ritorni logaritmici mensili di BAC/JPM per l'anno 2020}
    \label{fig:bac_jpm_log_return_2020}
  \end{minipage}%
  \begin{minipage}{.6\textwidth}
    \centering
    \includegraphics[width=1\linewidth]{bac_jpm_corr.png}
    \captionof{figure}{Correlazione nel tempo di BAC/JPM}
    \label{fig:bac_jpm_corr}
  \end{minipage}
\end{figure}

Dal grafico a figura \ref{fig:bac_jpm_corr} si nota come la correlazione tra i due titoli sia sempre stata piuttosto alta, con valori mai sotto \verb|0.6|.
Dal 2015 viene evidenziato una notevole crescita nella correlazione che dopo il 2016 non raggiungerà valori inferiori a \verb|0.85|.

\begin{figure}[ht]
  \centering
  \includegraphics[width=0.6\textwidth]{bac_jpm_disp.png}
  \caption{Dispersione rendimenti BAC/JPM}
  \label{fig:bac_jpm_disp}
\end{figure}

Dallo scatter plot mostrato a figura \ref{fig:bac_jpm_disp} viene confermata la forte correlazione.

\pagebreak