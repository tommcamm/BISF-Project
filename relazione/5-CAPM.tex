%%% CAPITOLO 5
%%% Capital Asset Pricin Model - CAPM - Valutazione del rischio
\section{Capital Asset Pricing Model - CAPM}

Il CAPM (\textbf{C}apital \textbf{A}sset \textbf{P}ricing \textbf{M}odel) è un modello che rappresenta la relazione tra 
i rendimenti aspettati di un indice rischioso ed il rischio di mercato (chiamato anche rischio sistematico).

Possiamo rappresentare il CAPM mediante la equazione:

\begin{displaymath}
    E(r_i) = r_f + \beta_i (E(r_m) - r_f)
\end{displaymath}

\textbf{dove}\\
\(E(r_i) = \text{Denota il rendimento aspettato dell'asset } i\)\\
\(r_f = \text{ratio risk-free}\)\\
\(E(r_m) = \text{Rendimento aspettato del mercato}\)\\
\(\beta = \text{coefficiente beta}\)

\subsection{Calcolo dell'indice beta}

L'indice beta si può interpretare come il livello di sensitività dei rendimenti di un asset relativamente al mercato.

\begin{itemize}
    \item \(\text{beta} \le -1\): L'asset si muove nella direzione opposta al benchmark e in un ammontare superiore rispetto al negativo del benchmark.
    \item \(-1 < \text{beta} < 0\): L'asset si muove nella direzione opposta al benchmark.
    \item \(\text{beta} = 0\): Non esiste correlazione tra il movimento del prezzo dell'asset e il benchmark di mercato.
    \item \(0 < \text{beta} < 1\): L'asset si muove nella stessa direzione del mercato, ma con un ammontare inferiore. (per esempio uno stock di una compagnia che non è molto suscettibile a fluttuazioni giornaliere).
    \item \(\text{beta} = 1\): L'asset ed il mercato si muovono nella stessa direzione e con lo stesso ammontare.
    \item \(\text{beta} \ge 1\): L'asset si muove nella stessa direzione del mercato, ma l'ammontare è maggiore (per esempio nel caso di stock che sono molto suscettibili a cambiamenti giornalieri nelle notizie di mercato). 
\end{itemize}

La equazione del CAPM può essere modificata per ottenere la formula di calcolo per beta

\begin{displaymath}
    \beta = \frac{cov(R_i,R_m)}{var(R_m)}
\end{displaymath}

Nel calcolo di beta per tutti i titoli considerati in questo progetto verrà considerato l'indice \verb|S&P 500| (\verb|^GSPC|).

Mediante il metodo delle covarianze l'indice beta per ogni titolo è:
\begin{itemize}
    \item Meta (FB): $1.1146$
    \item Alphabet (GOOG): $1.0138$
    \item Raytheon (RTX): $1.3118$
    \item Lockheed Martin(LMT): $0.835$
    \item Bank of America(BAC): $1.5124$
    \item JPMorgan Chase (JPM): $1.2579$
\end{itemize}

\pagebreak

\subsection{Calcolo esposizione con Fama-French}

